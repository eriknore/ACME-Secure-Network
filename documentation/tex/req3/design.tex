\section*{Design}

Here we present our design categorized under the different types of security addressing all parts of the requirements.

\subsection*{Confidentiality}

All traffic between the London and Stockholm network is encrypted using a Virtual Private Network (VPN) in tunnel mode (using SSL), the software used is \href{http://openvpn.net/}{OpenVPN}. All traffic between employees outside of any of the networks using the web server is encrypted using \href{http://www.openssl.org/}{OpenSSL} after proper authentication has been performed (two-factor auth, more details below).

\subsubsection*{Perimeter Security}

A Firewall is placed at the entry point of each network using stateful packet filters. This will prevent unauthorized outsiders on the internet to get access to the internal network. The firewall is implemented using \href{http://en.wikipedia.org/wiki/Iptables}{iptables} and will also be used for routing.

All traffic going to and coming from the webserver will be logged. The webserver will be an HTTPS \href{http://httpd.apache.org/}{Apache} server.

\subsubsection*{Internal Security}

It is assumed that the internal network is physically secure from outsiders, i.e. that an outsider does not have physical access to the offices of ACME. In other words there is no need to encrypt the internal traffic.

An open source IDS called \href{https://www.snort.org/}{Snort} is installed at the Stockholm office monitoring traffic and raising an alarm if intrusion is detected. It will use both anomaly and signature detection.w

\subsection*{Authentication}

All authentication will be performed using private and public keys issued from an internal Public key infrastructure (PKI), implemented using \href{https://pki.openca.org/}{OpenSSL}. The wireless network will use WPA2 Enterprise (IEEE 802.11i) and EAP-TLS with a RADIUS server to authenticate users using digital certificates.

To access resources on the web server a 2FA or 3FA can be used depending on if employees are required to have their certificates or not. If that is the case a two-way authentication can be made, meaning both client and server are authenticated using certificates issued by ACMEs CA. This is what we would recommend. If it is not possible a one-way authentication of the server will be made and all traffic will be encrypted using SSL. Either way a user account is generated for each emplyee. These can be protected by both a static password/PIN-code as well as a 6-digit one time password generated by the mobile application \href{http://en.wikipedia.org/wiki/Google_Authenticator}{Google Authenticator}, or only an OTP if that would be preffered. The OTP is regenerated every 30 seconds and after 5 bad login attempts the account is locked.

\subsection*{Additional services}

There was also a requirement that it would be possible to transfer files between company phones. This will be solved by encrypting the file using the recpient's public key achieving confidentiality and integrity. The file is transfered over HTTPS after verification of certificates to the server. Lastly the receiver is notified and can retrieve the file for decryption. We will host a small API on the web server which will be used by the android application.