\section*{Design}

Here we present our design categorized under the different types of security addressing all parts of the requirements.

\subsection*{Confidentiality}

All traffic between the London and Stockholm network is encrypted using a Virtual Private Network (VPN) in tunnel mode (using SSL), the software used is \href{http://openvpn.net/}{OpenVPN}. All traffic between employees outside of any of the networks using the web server is encrypted using \href{http://www.openssl.org/}{OpenSSL} after proper authentication has been performed (two-factor auth, more details below).

\subsubsection*{Perimeter Security}

A Firewall is placed at the entry point of each network using stateful packet filters. This will prevent unauthorized outsiders on the internet to get access to the internal network. The firewall is implemented using \href{http://en.wikipedia.org/wiki/Iptables}{iptables}.

All traffic going to and coming from the webserver will be logged. The webserver will be an HTTPS \href{http://httpd.apache.org/}{Apache} server.

\subsubsection*{Internal Security}

It is assumed that the internal network is physically secure from outsiders, i.e. that an outsider does not have physical access to the offices of ACME. In other words there is no need to encrypt the internal traffic.

An open source IDS called \href{https://www.snort.org/}{Snort} is installed at the Stockholm office monitoring traffic and raising an alarm if intrusion is detected. It will use both anomaly and signature detection.

\subsection*{Authentication}

All authentication will be performed using private and public keys issued from an internal Public key infrastructure (PKI), implemented using \href{https://pki.openca.org/}{OpenCA}. The wireless network will use WPA2 Enterprise (IEEE 802.11i) and EAP-TLS with a RADIUS server to authenticate users using digital certificates.

To access resources on the web server an employee must identify her-/himself using the digital certificate, however also use an application on the company mobile phone to further identify her- or himself. The mobile application for this will be \href{http://en.wikipedia.org/wiki/Google_Authenticator}{Google Authenticator}.
